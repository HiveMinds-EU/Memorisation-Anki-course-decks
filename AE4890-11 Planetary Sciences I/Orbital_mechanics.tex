% -*- coding:utf-8 -*-
\documentclass[12pt]{article}
\usepackage[utf8]{inputenc}
\newenvironment{note}{\paragraph{NOTE:}}{}
\newenvironment{field}{\paragraph{field:}}{}
\begin{document}
\begin{note}
   \begin{field}
        What is Shepherding?
   \end{field}
   \begin{field}
		It is when two moons keep a (braided) ring of particles between them.
   \end{field}
\end{note}
%%%
\begin{note}
   \begin{field}
        How does Shepherding work?
   \end{field}
   \begin{field}
		If a outer moon passes an inner ring particle, the particle goes slower than the moon, so it gets dragged down by the moon, which lowers its velocity and hence altitude. An outer partcle gets accellerated to a higher orbit. So if two moons do this with a string of particles in between, the lower moon pushes the particles up, and the upper moon pushes the particles down, creating a shepherded ring of particles between the two moons.
   \end{field}
\end{note}
%%%
\begin{note}
   \begin{field}
        Draw the lagrange points
   \end{field}
   \begin{field}
   		   4
		3,1,2
		   5
   \end{field}
\end{note}
%%%
\begin{note}
   \begin{field}
        Whats the orbit name around L4?
   \end{field}
   \begin{field}
   		   Tadpole
   \end{field}
\end{note}
%%%
\begin{note}
   \begin{field}
        Whats the orbit name around L2?
   \end{field}
   \begin{field}
   		   Halo
   \end{field}
\end{note}
%%%
\begin{note}
   \begin{field}
        Why are there two tides on Earth? (Start numbering like: 0. reason.)
   \end{field}
   \begin{field}
		0. Moon pullling a tide.
		1. Centrifugal force of Moon-Earth rotating around same barycenter.
   \end{field}
\end{note}
\end{document}