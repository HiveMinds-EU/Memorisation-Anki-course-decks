% -*- coding:utf-8 -*-
\documentclass[12pt]{article}
\usepackage[utf8]{inputenc}
\newenvironment{note}{\paragraph{NOTE:}}{}
\newenvironment{field}{\paragraph{field:}}{}
\begin{document}
\section{Solar System Dynamics}
\begin{note}
   \begin{field}
       Give the equation for the attraction between two bodies Sun and Earth, commaseparate the equation with its units. (using frac if necessary).
   \end{field}
   \begin{field}
		\[
			F=G\frac{M_{Sun}M_{Earth}}{r^2}
		\],
		\[
			N
		\]
   \end{field}
\end{note}
%%%
\begin{note}
   \begin{field}
       Give the Vis-Viva equation, commaseparate the equation with its units. (using frac if necessary).
   \end{field}
   \begin{field}
		\[
			\frac{V^2}{2}=\frac{\mu}{r}-\frac{\mu}{2a}
		\],
		\[
			\frac{m^2}{s^2}
		\]
   \end{field}
\end{note}
%%%
\begin{note}
   \begin{field}
       Give the equation for orbital period, commaseparate the equation with its units. (using frac if necessary).
   \end{field}
   \begin{field}
		\[
			T=2\pi\sqrt{\frac{r^3}{\mu}}
		\],
		\[
			s
		\]
   \end{field}
\end{note}
%%%
\begin{note}
   \begin{field}
       Give the equation for escape velocity\(V_{esc}\), commaseparate the equation with its units. (using frac if necessary).
   \end{field}
   \begin{field}
		\[
			V_{esc}=\sqrt{\frac{2\mu}{r}}
		\],
		\[
			\frac{m}{s}
		\]
   \end{field}
\end{note}
%%%
\begin{note}
   \begin{field}
       Give the Hill equation for Jupiter and the Sun, commaseparate the equation with its units. (using frac if necessary).
   \end{field}
   \begin{field}
		\[
			r_{hill}=a{\left(\frac{m_{Jupiter}}{3(m_{Sun}+m_{Jupiter}}\right)}^{\frac{1}{3}}
		\],
		\[
			m
		\]
   \end{field}
\end{note}
\section{Minor Bodies and Comets}
\begin{note}
   \begin{field}
       Give the equation for the minimum radius of a spherical body, commaseparate the equation with its units. (using frac if necessary).
   \end{field}
   \begin{field}
		\[
			R_{min}=\sqrt{\frac{2S}{\pi G \rho^2}}
		\],
		\[
			m
		\]
   \end{field}
\end{note}
%%%
\begin{note}
   \begin{field}
       Give the equation to compute the gravitational constant of a planet, commaseparate the equation with its units. (using frac if necessary).
   \end{field}
   \begin{field}
		\[
			\mu=MG
		\],
		\[
			\frac{{km}^3}{s^2}
		\]
   \end{field}
\end{note}
\begin{note}
   \begin{field}
       Give the equation to compute the spherical harmonics of Earth due to the Moon, commaseparate the equation with its units. (using frac if necessary).
   \end{field}
   \begin{field}
		\[
			U=\frac{\mu_{Moon}}{R_{Earth-Moon}}\sum_{n=2}^\infty {\left(\frac{R_{Earth}}{R_{Earth-Moon}}\right)}^n P_n \cos{\phi}
		\],
		\[
			\frac{{m}^2}{s^2}
		\]
   \end{field}
\end{note}
\begin{note}
   \begin{field}
       Give the equation to compute the legendre polynomials, ommit units.
   \end{field}
   \begin{field}
		\[
			P_n(x)=\frac{1}{2^n n!}\frac{d^n}{dx^n}{\left(x^2-1\right)}^n
		\],
		\[
			\frac{{m}^2}{s^2}
		\]
   \end{field}
\end{note}
%%%
\begin{note}
   \begin{field}
       Give the 0th,1st and 2nd legendre polynomials comma separated, ommit units.
   \end{field}
   \begin{field}
		\[
			P_0(x)=1
		\],
		\[
			P_1(x)=x
		\],
		\[
			P_2(x)=\frac{1}{2}(3x^2-1)
		\]
   \end{field}
\end{note}
%%%
\begin{note}
   \begin{field}
       Give the equation for equilibrium temperatures.
   \end{field}
   \begin{field}
		\[
			T = \frac{1-A_B}{4\epsilon\sigma}\frac{F}{d^2}
		\]
   \end{field}
\end{note}

%%%
\begin{note}
   \begin{field}
       Give the equation for Roches limit between a planet and its sat(telite). Assume it should not break at the surface.
   \end{field}
   \begin{field}
		\[
			\frac{3\mu_{planet}}{d_{planet-sat}}r_{sat}=\frac{\mu_{sat}}{r_{sat}^2}
		\]
   \end{field}
\end{note}
%%%
\begin{note}
   \begin{field}
       Give the equation for Roches limit between a planet and its sat(telite). Assume it should not break at the tidal pull \(F_g=F_t\).
   \end{field}
   \begin{field}
		\[
			\frac{Gm_{sat}\mu_{sat}}{r_{sat}^2}=\frac{2Gm_{sat}\mu_{sat}r_{sat}}{d_{planet-sat}^3}
		\]
   \end{field}
\end{note}
%%%
\begin{note}
   \begin{field}
       Give the solar flux arriving at Earth, with its value and units. (using frac if necessary).
   \end{field}
   \begin{field}
		\[
			F=\frac{L}{4\pi d_{Sun-Earth}}=1366
		\],
		\[
			\frac{W}{m^2}
		\]
   \end{field}
\end{note}
\end{document}